\documentclass[final,xcolor=pst,dvips]{beamer}
%\documentclass[final]{beamer}
\usepackage{beamerthemeshadow} % items as shadowed circles
%\usepackage{pstricks,pst-grad,pst-text,pst-char}
\usepackage{pstricks,pst-text,pst-char,pst-grad,multido}
\usepackage{pst-node}
\usepackage{avant} % font
\usepackage{animate,multirow,rotating}
\usepackage{tcolorbox,hyperref,setspace}
\renewcommand{\familydefault}{\sfdefault}
\usepackage{amsmath,sfmath,url}

\usepackage{tikz}
\usetikzlibrary{calc}

\setbeamercolor{background canvas}{bg=black}
\setbeamercolor{normal text}{fg=white,bg=black!90}

\setbeamertemplate{navigation symbols}{} 
\definecolor{zhawblue}{HTML}{1166AD}
\setbeamercolor{leftcolour}{bg=zhawblue,fg=white}
\setbeamercolor{rightcolour}{bg=gray,fg=black}

\definecolor{lightblue}{RGB}{51,255,255}
\definecolor{midblue}{RGB}{0,102,204}
\definecolor{gray05}{gray}{0.05}
\definecolor{gray10}{gray}{0.1}
\definecolor{gray20}{gray}{0.2}
\definecolor{gray50}{gray}{0.5}
\definecolor{gray80}{gray}{0.8}
\definecolor{gray90}{gray}{0.9}
\definecolor{PaleGreen}{RGB}{51,255,51}
\definecolor{PaleGreenLow}{RGB}{13,64,13}
\definecolor{ForestGreen}{RGB}{51,102,0}
\definecolor{BrightOrange}{RGB}{255,127,0}
\definecolor{BrightOrangeLow}{RGB}{96,48,0}
\definecolor{MidRed}{RGB}{200,50,0}
\definecolor{white}{RGB}{255,255,255}

\DeclareFixedFont{\mfa}{T1}{phv}{b}{n}{0.5cm}
\DeclareFixedFont{\mfb}{T1}{phv}{b}{n}{0.7cm}
\DeclareFixedFont{\mfc}{T1}{phv}{b}{n}{0.8cm}
\DeclareFixedFont{\mfd}{T1}{phv}{b}{n}{1.0cm}
\DeclareFixedFont{\mfe}{T1}{phv}{b}{n}{1.1cm}
\DeclareFixedFont{\mff}{T1}{phv}{b}{n}{1.2cm}
\DeclareFixedFont{\mfg}{T1}{phv}{b}{n}{1.3cm}
\DeclareFixedFont{\mfh}{T1}{phv}{b}{n}{2.0cm}

\setbeamertemplate{headline}
{
    \leavevmode%
    \hbox{%
        \begin{beamercolorbox}[wd=0.5\paperwidth,ht=2.25ex,dp=1ex,left]{leftcolour}%
            \usebeamerfont{subsection in head/foot}\hspace*{2ex}\inserttitle
        \end{beamercolorbox}\hspace*{0.5pt}}%
    \hbox{%
        \begin{beamercolorbox}[wd=0.5\paperwidth,ht=2.25ex,dp=1ex,right,rightskip=1em]{rightcolour}%
            \usebeamerfont{subsection in head/foot}\hspace*{2ex}today's date\hspace*{2em} 
            %\insertframenumber{} / \inserttotalframenumber\hspace*{2ex}
        \end{beamercolorbox}}%
    \vskip0pt%
}
\setbeamertemplate{footline}{}

% ******** progressbar in footline ********
% http://tex.stackexchange.com/questions/59742/progress-bar-for-latex-beamer
\makeatletter
\def\progressbar@progressbar{} % the progress bar
\newcount\progressbar@tmpcounta% auxiliary counter
\newcount\progressbar@tmpcountb% auxiliary counter
\newdimen\progressbar@pbht %progressbar height
\newdimen\progressbar@pbwd %progressbar width
\newdimen\progressbar@rcircle % radius for the circle
\newdimen\progressbar@tmpdim % auxiliary dimension

\progressbar@pbwd=\linewidth
\progressbar@pbht=1pt
\progressbar@rcircle=2.5pt

% total pages for this version
\newcounter{totalpages}
\setcounter{totalpages}{7}

\def\progressbar@progressbar{%

    \progressbar@tmpcounta=\insertframenumber
    %\progressbar@tmpcountb=\inserttotalframenumber
    \progressbar@tmpcountb=\arabic{totalpages}
    \progressbar@tmpdim=\progressbar@pbwd
    \multiply\progressbar@tmpdim by \progressbar@tmpcounta
    \divide\progressbar@tmpdim by \progressbar@tmpcountb

    \begin{tikzpicture}
        \draw[zhawblue!30,line width=\progressbar@pbht]
        (0pt, 0pt) -- ++ (\progressbar@pbwd,0pt);

        \filldraw[zhawblue!30] %
            (\the\dimexpr\progressbar@tmpdim-\progressbar@rcircle\relax, .5\progressbar@pbht) circle (\progressbar@rcircle);

        %\node[draw=zhawblue!30,text width=3.5em,align=center,inner sep=1pt,
        %    text=zhawblue!70,anchor=east] at (0,0) {\insertframenumber/\inserttotalframenumber};
        %\node[draw=zhawblue!30,text width=3.5em,align=center,inner sep=1pt,
        %    text=zhawblue!70,anchor=east] at (0,0) {\insertframenumber/\arabic{totalpages}};
    \end{tikzpicture}%
}

\addtobeamertemplate{footline}{
    {%
      \begin{beamercolorbox}[wd=\paperwidth,ht=4ex,center,dp=1ex]{white}%
          \progressbar@progressbar%
        \end{beamercolorbox}%
    }
}
\makeatother
% ******** end progressbar ********

\newcommand{\ball}{{\usebeamertemplate{itemize item}\hspace{2mm}}}

\usepackage{listings}
\lstset{ %
    language=R,                     % the language of the code
    basicstyle=\scriptsize\tt\color{gray90},      % the size of the fonts that are used for the code
    numbers=left,                   % where to put the line-numbers
    numberstyle=\tiny\color{gray},  % the style that is used for the line-numbers
    stepnumber=1,                   % the step between two line-numbers. If it's 1, each line will be numbered
    numbersep=5pt,                  % how far the line-numbers are from the code
    backgroundcolor=\color{gray50}, % choose the background color. You must add \usepackage{color}
    showspaces=false,               % show spaces adding particular underscores
    showstringspaces=false,         % underline spaces within strings
    showtabs=false,                 % show tabs within strings adding particular underscores
    frame=single,                   % adds a frame around the code
    rulecolor=\color{black},        % if not set, the frame-color may be changed on line-breaks within not-black text (e.g. commens (green here))
    tabsize=2,                      % sets default tabsize to 2 spaces
    captionpos=b,                   % sets the caption-position to bottom
    breaklines=false,               % sets automatic line breaking
    breakatwhitespace=false,        % sets if automatic breaks should only happen at whitespace
    title=\lstname,                 % show the filename of files included with \lstinputlisting; also try caption instead of title
    keywordstyle=\color{yellow},    % keyword style
    commentstyle=\color{PaleGreen},   % comment style
    stringstyle=\color{BrightOrange},    % string literal style
    escapeinside={\%*}{*)},         % if you want to add a comment within your code
    morekeywords={*,\ldots},        % if you want to add more keywords to the set
    %escapeinside=||,
    alsoletter={_},
    otherkeywords={!,!=,~,$,*,\&,\%/\%,\%*\%,\%\%,<-,<<-,\%>\%,::,%
        key,value,bg,border,size,value_exact},
    moredelim=[is][\yellow]{|}{|}
}

\lstset{emph={%  
    opq,add_feature,osmdata_sf,osmdata,get_bbox,osmplotr,osm_basemap,add_osm_objects,getbb,%
    print_osm_map,extract_osm_objects,adjust_colours,add_axes,add_osm_surface, get_surface_data,%
    RColorBrewer,brewer,pal,add_colourbar,head%
    },emphstyle={\color{green}\bfseries}%
}%

\linespread{1.2}

\title[short title]{this is the title}
\author[short author]{this is the author}


\begin{document}

\setcounter{framenumber}{2}

\section[SecTwo]{Autocorrelation}

{\setbeamercolor{background canvas}{bg=gray05} 
\begin{frame}
    \begin{center}
        \begin{pspicture}(0,0)(20,8)
            \rput[bl](1.5,0){ \includegraphics[height=0.8\paperheight]{./figures/fig1.eps}}
            \pause
            \rput[l](3.2,6.8){\color{white}{\large $\Delta H\sim\Delta D^b$}}
            \pause
            \psline[linecolor=gray,linewidth=5pt]{->}(1.8,4)(2.8,4)
            \psline[linecolor=gray80,linewidth=3pt]{->}(1.85,4)(2.7,4)
            \rput[l](3.0,4){\pscharpath[linewidth=1pt,linecolor=gray,fillstyle=gradient,gradmidpoint=0.5,
                gradbegin=gray,gradend=white] {\mfd A fancy equation}}
        \end{pspicture}
    \end{center}
\end{frame}
}

{\setbeamercolor{background canvas}{bg=gray05} 
\begin{frame}
    \begin{center}
        \begin{pspicture}(0,0)(10,7)
            \rput[c](5,6){\pscharpath[linewidth=1pt,linecolor=gray,fillstyle=gradient,gradmidpoint=0.5,
                gradbegin=white,gradend=gray] {\mfb Here is a lovely}}
            \rput[c](5,5){\pscharpath[linewidth=1pt,linecolor=gray,fillstyle=gradient,gradmidpoint=0.5,
                gradbegin=white,gradend=gray] {\mfb bunch of text}}
            \rput[c](5,4){\pscharpath[linewidth=1pt,linecolor=gray,fillstyle=gradient,gradmidpoint=0.5,
                gradbegin=white,gradend=gray] {\mfb which looks}}
            \rput[c](5,3){\pscharpath[linewidth=1pt,linecolor=gray,fillstyle=gradient,gradmidpoint=0.5,
                gradbegin=white,gradend=gray] {\mfb quite nice}}
            \pause

            \rput[l](1,2){\color{white}{And here is some more text}}
            \rput[l](1,1.5){\color{white}{in the default font}}
            \rput[l](1,1){\color{white}{which looks kind of okay}}
            \pause

            \rput[c]{10}(5,3){\pscharpath[linewidth=1pt,linecolor=gray,fillstyle=gradient,gradmidpoint=0.5,
                                    gradbegin=lightblue,gradend=midblue] {\mfe diagonal}}
            \pause

            \rput[c]{-10}(5,3){\pscharpath[linewidth=1pt,linecolor=gray,fillstyle=gradient,gradmidpoint=0.5,
                                    gradbegin=PaleGreen,gradend=ForestGreen] {\mfe the other diagonal}}
        \end{pspicture}
    \end{center}
\end{frame}
}

{\setbeamercolor{background canvas}{bg=gray05} 
\begin{frame}
    \begin{center}
        \begin{pspicture}(0,0)(20,7)
            \rput[c](5,6.3){\large \color{white}{$\Delta Y=f(\Delta d)$}}
            \onslide<1>{
                \rput[bc](5,3){ \includegraphics[height=0.6\paperheight]{./figures/fig2a.eps}}
            }
            \onslide<2>{
                \rput[bc](5,3){ \includegraphics[height=0.6\paperheight]{./figures/fig2b.eps}}
            }
            \onslide<3>{
                \rput[bc](5,3){ \includegraphics[height=0.6\paperheight]{./figures/fig2c.eps}}
            }
            \onslide<4>{
                \rput[bc](5,3){ \includegraphics[height=0.6\paperheight]{./figures/fig2d.eps}}
            }
            \onslide<5->{
                \rput[bc](5,3){ \includegraphics[height=0.6\paperheight]{./figures/fig2e.eps}}
                \psline[linecolor=midblue,linewidth=1pt]{-}(6.6,0.1)(6.6,0.0)(7.6,0.0)(7.6,0.1)
                \rput(7.1,-0.2){\color{midblue}{2}}
            }
        \end{pspicture}
    \end{center}
\end{frame}
}

\end{document}
